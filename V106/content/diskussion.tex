\section{Diskussion}
\label{sec:Diskussion}
Für alle Messwerte wurde die Ungenauigkeit nur als Ableseungenauigkeit
angegeben, die tatsächliche Ungenauigkeit liegt also höher, lässt sich allerdings
nur schwer abschätzen. Die wichtigsten Faktoren dabei sind Fehler bei der
Messung und Durchführung die kaum zu vermeiden sind. So ist die Pendelbewegung
z.B. nicht rein 2 dimensional, stattdessen schwingen die Pendel geringfügig
senkrecht zu der erwünschten Richtung. Dadurch entstehen weitere Abweichungen
bei der Schwingungs- und Schwebungsdauer. Außerdem ist die Genauigkeit der
Messung durch die Reaktionsgeschwindigkeit beim Drücken begrenzt.

\subsection{Schwingungsdauern ungekoppelt und gleichphasig}
Die gemessenen Schwingungsdauern weichen für beide Pendellängen nur
geringfügig voneinander ab. Die geringen Abweichungen lassen sich durch
kleine Fehler bei der Positionierung der Massen erklären.
Dazu kommt, dass die Messungen abwechselnd von zwei Personen durchgeführt wurden,
wodurch wegen unterschiedlicher Reaktionszeiten erneut Abweichungen dazukommen.
Bei beiden Pendellängen weicht der Theoriewert kaum von den gemessenen Werten ab,
was dafür spricht, dass der Aufbau sein Ziel der Reibungsminimierung
zufriedenstellend erfüllt.

\subsection{Gegenphasige und gekoppelte Schwingung}
Die Werte für die gegenphasige Schwingung liegen erwartungsgemäß deutlich
niedriger als bei der gleichphasigen. Da der Kopplunggrad nicht bekannt ist
(was hier als Kopplungsgrad verwendet wird, ergibt sich aus $T_+$ und $T_-$ und
hilft daher nicht $T_-$ einzuordnen), ist es nicht möglich Aussagen über
den absoluten Wert von $T_-$ zu treffen.
Die Schwingungsdauer der gekoppelten Schwingung ist ebenfalls schwierig
absolut einzuordnen. Aus der Anleitung ist nur bekannt, dass die gekoppelte
Schwingung als Überlagerung der Eigenschwingungen zu betrachten ist.
Daher scheinen die Messwerte in das Gesamtbild zu passen. Eine genauere
Einordnung wird auch dadurch erschwert, dass die gemessene gekoppelte Schwingung
nicht exakt der theoretischen entspricht. In der Praxis bleibt das zweite
Pendel nicht bei $\phi=0$ in Ruhe, wenn das erste ausgelenkt wird. Stattdessen
ergibt sich für dieses Pendel aufgrund der Federkraft eine neue Ruhelage bei
$\phi \neq 0$.

\subsection{Schwebungsdauer}


Bei der Schwebungsdauer der gekoppelten Schwingung fällt auf, dass
(für beide Pendellängen) der gemessene (und gemittelte) Wert recht deutlich
($\approx$ 15/18\%) von dem Theoriewert ab, der sich aus $T_+$ und $T_-$ ergibt.
Dabei sein anzumerken, dass die Bestimmung der Schwebungsdauer mit
den vorhandenen Mitteln nur begrenzt
genau möglich ist, da der Zeitpunkt an dem das Pendel still steht schwierig
zu messen ist. Teilweise bleibt das Pendel auch über einen gewissen Zeitraum
(annähernd) in Ruhe, was die Messung weiter erschwert.
