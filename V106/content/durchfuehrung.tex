\section{Aufbau und Durchführung}
\label{sec:Durchführung}

Als Pendel werden zwei identische Stabpendel mit einer
Spitzenlagerung verwendet.
Die jeweils zwei Spitzen der Pendel hängen in
einer keilförmigen Nut um die Reibung zu minimieren. Mit einer Feder werden
die Pendel gekoppelt. Die Pendelmassen ($m = \si{1}{kg})
sind entlang des Pendels verschiebbar.
Auf diese Weise lässt sich die Pendellänge einstellen.


Für die Messungen werden beide Pendel auf dieselbe Länge $l_1$
eingestellt und zur 
Kontrolle die Gleichheit der Schwingungsdauern überprüft. Die Pendel
sind dabei nicht gekoppelt. Für die Messung von Schwingungsdauern werden
fünf Perioden gemessen und der Mittelwert gebildet.
Um den Einfluss von Messfehlern
zu minimieren, werden alle Messungen außerdem jeweils zehnmal durchgeführt.
Daraufhin werden die Pendel gekoppelt und die Schwingungsdauern des
Doppelpendels bei gleich- und gegenphasiger Schwingung gemessen.
Für die letzte Messreihe wird nur ein Pendel ausgelenkt, während das andere
bei $t = \si{0}{s}$ in Nulllage hängt. Gemessen werden die Schwingungs- und
Schwebungsdauer dieser Schwingugsform. Für die Schwebungsdauer wird nur eine
halbe Periode betrachtet.

Die gesammte Messreihe wird mit einer weiteren Pendellänge $l_2$ wiederholt.
