\documentclass[
  bibliography=totoc,     % Literatur im Inhaltsverzeichnis
  captions=tableheading,  % Tabellenüberschriften
  titlepage=firstiscover, % Titelseite ist Deckblatt
]{scrartcl}

% Paket float verbessern
\usepackage{scrhack}

% Warnung, falls nochmal kompiliert werden muss
\usepackage[aux]{rerunfilecheck}

% deutsche Spracheinstellungen
\usepackage{polyglossia}
\setmainlanguage{german}

% unverzichtbare Mathe-Befehle
\usepackage{amsmath}
% viele Mathe-Symbole
\usepackage{amssymb}
% Erweiterungen für amsmath
\usepackage{mathtools}

% Fonteinstellungen
\usepackage{fontspec}
% Latin Modern Fonts werden automatisch geladen

\usepackage[
  math-style=ISO,    % ┐
  bold-style=ISO,    % │
  sans-style=italic, % │ ISO-Standard folgen
  nabla=upright,     % │
  partial=upright,   % ┘
  warnings-off={           % ┐
    mathtools-colon,       % │ unnötige Warnungen ausschalten
    mathtools-overbracket, % │
  },                       % ┘
]{unicode-math}

% traditionelle Fonts für Mathematik
\setmathfont{Latin Modern Math}
\setmathfont{XITS Math}[range={scr, bfscr}]
\setmathfont{XITS Math}[range={cal, bfcal}, StylisticSet=1]

% Zahlen und Einheiten
\usepackage[
  locale=DE,                 % deutsche Einstellungen
  separate-uncertainty=true, % immer Fehler mit \pm
  per-mode=reciprocal,       % ^-1 für inverse Einheiten
  output-decimal-marker=.,   % . statt , für Dezimalzahlen
]{siunitx}

% chemische Formeln
\usepackage[
  version=4,
  math-greek=default, % ┐ mit unicode-math zusammenarbeiten
  text-greek=default, % ┘
]{mhchem}

% richtige Anführungszeichen
\usepackage[autostyle]{csquotes}

% schöne Brüche im Text
\usepackage{xfrac}

% Standardplatzierung für Floats einstellen
\usepackage{float}
\floatplacement{figure}{htbp}
\floatplacement{table}{htbp}

% Floats innerhalb einer Section halten
\usepackage[
  section, % Floats innerhalb der Section halten
  below,   % unterhalb der Section aber auf der selben Seite ist ok
]{placeins}

% Seite drehen für breite Tabellen
\usepackage{pdflscape}

% Captions schöner machen.
\usepackage[
  labelfont=bf,        % Tabelle x: Abbildung y: ist jetzt fett
  font=small,          % Schrift etwas kleiner als Dokument
  width=0.9\textwidth, % maximale Breite einer Caption schmaler
]{caption}
% subfigure, subtable, subref
\usepackage{subcaption}

% Grafiken können eingebunden werden
\usepackage{graphicx}
% größere Variation von Dateinamen möglich
\usepackage{grffile}

% schöne Tabellen
\usepackage{booktabs}

% Verbesserungen am Schriftbild
\usepackage{microtype}

% Literaturverzeichnis
\usepackage[
  backend=biber,
]{biblatex}
% Quellendatenbank
\addbibresource{lit.bib}
\addbibresource{programme.bib}

% Hyperlinks im Dokument
\usepackage[
  unicode,        % Unicode in PDF-Attributen erlauben
  pdfusetitle,    % Titel, Autoren und Datum als PDF-Attribute
  pdfcreator={},  % ┐ PDF-Attribute säubern
  pdfproducer={}, % ┘
]{hyperref}
% erweiterte Bookmarks im PDF
\usepackage{bookmark}

% Trennung von Wörtern mit Strichen
\usepackage[shortcuts]{extdash}

\author{
  Lukas Nickel
  \texorpdfstring{
    \\
    \href{mailto:lukas.nickel@tu-dortmund.de}{lukas.nickel@tu-dortmund.de}
  }{}%
  \texorpdfstring{\and}{, }
  Rohat Kavili
  \texorpdfstring{
    \\
    \href{mailto:rohat.kavili@tu-dortmund.de}{rohat.kavili@tu-dortmund.de}
  }{}%
}
\publishers{TU Dortmund – Fakultät Physik}

\subject{Versuch 106}
\title{Gekoppelte Pendel - Zur Fehlerberechnung}
\date{
  Durchführung: 01.12.2015
  \hspace{3em}
  Abgabe: 05.01.2016
}
\begin{document}
\thispagestyle{empty}
Für die Berechnung der Fehler wurde die Gauß'sche Fehlerfortpflanzung
verwendet. Danach gilt für den Fehler $u_y$ einer resultierenden Größe $y$
abhängig von mehreren, unabhängigen, fehlerbehafteten Variablen $x_\text{i}$:
\begin{equation*}
  \centering
  \text{\Delta} y = \sqrt{\sum{(\frac{\partial y}{\partial x_\text{i}})\cdot u_{\text{i}}}}.
  \label{eqn:gauss}
\end{equation*}

Für den Fehler der Schwingungsdauer $T_+ = 2 \pi\sqrt{l/g}$ der gleichsinnigen
Schwingung gilt dann:
\begin{equation*}
\sigma_{T_+}=\sqrt{\frac{\pi^{2} \sigma_{l}^{2}}{g l}}.
\end{equation*}
Die Schwingungsdauer $$T_- = \frac{2\pi}{\sqrt{\frac{g}{l}+\frac{2K}{l}}}$$
 der gegensinnigen Schwingung wurde nur gemessen, da
der Kopplungsgrad $K$ nur aus $T_+$ und $T_-$ bekannt ist.
Damit entfällt die Fehlerrechnung für den Theoriewert.

Der Fehler des Kopplungsgrades $K=\frac{T_+^2-T_-^2}{T_+^2+T_-^2}$ errechnet sich
 zu:
\begin{equation*}
\sigma_K =
\sqrt{\sigma_{T_{+}}^{2} \left(- \frac{2 T_{+} \left(T_{+}^{2} -
T_{-}^{2}\right)}{\left(T_{+}^{2} + T_{-}^{2}\right)^{2}} +
\frac{2 T_{+}}{T_{+}^{2} + T_{-}^{2}}\right)^{2} + \sigma_{T_{-}}^{2}
\left(- \frac{2 T_{-} \left(T_{+}^{2} - T_{-}^{2}\right)}{\left(T_{+}^{2} +
T_{-}^{2}\right)^{2}} - \frac{2 T_{-}}{T_{+}^{2} + T_{-}^{2}}\right)^{2}}.
\end{equation*}

Für den Mittelwert gilt :
\begin{equation*}
\bar{x} = \frac{1}{n}\sum_{i=0}^{n}{x_\text{i}}.
\end{equation*}
womit sich der mittlere Fehler des Mittelwertes zu
$$\text{\Delta} \bar{x} = \frac{\sigma_x}{\sqrt{n}}$$
ergibt.

Die Fehlerformel für die Schwebungsdauer $$T_\text{Schwebung} = \frac
{T_+ \cdot T_-}{T_+-T_-} $$ist
\begin{equation*}
\sigma_{T_{\text{Schwebung}}}=
  \sqrt{\sigma_{T_{+}}^{2} \left(- \frac{T_{+} T_{-}}{\left(T_{+} -
  T_{-}\right)^{2}} + \frac{T_{-}}{T_{+} - T_{-}}\right)^{2} + \sigma_{T_{-}}^{2}
   \left(\frac{T_{+} T_{-}}{\left(T_{+} - T_{-}\right)^{2}} + \frac{T_{+}}{T_{+} - T_{-}}\right)^{2}}.
\end{equation*}
\end{document}
