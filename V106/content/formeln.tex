\input{header.tex}
\subject{Versuch 106}
\title{Gekoppelte Pendel - Zur Fehlerberechnung}
\date{
  Durchführung: 01.12.2015
  \hspace{3em}
  Abgabe: 05.01.2016
}
\begin{document}
\thispagestyle{empty}
Für die Berechnung der Fehler wurde die Gauß'sche Fehlerfortpflanzung
verwendet. Danach gilt für den Fehler $u_y$ einer resultierenden Größe $y$
abhängig von mehreren, unabhängigen, fehlerbehafteten Variablen $x_\text{i}$:
\begin{equation*}
  \centering
  \text{\Delta} y = \sqrt{\sum{(\frac{\partial y}{\partial x_\text{i}})\cdot u_{\text{i}}}}.
  \label{eqn:gauss}
\end{equation*}

Für den Fehler der Schwingungsdauer $T_+ = 2 \pi\sqrt{l/g}$ der gleichsinnigen
Schwingung gilt dann:
\begin{equation*}
\sigma_{T_+}=\sqrt{\frac{\pi^{2} \sigma_{l}^{2}}{g l}}.
\end{equation*}
Die Schwingungsdauer $$T_- = \frac{2\pi}{\sqrt{\frac{g}{l}+\frac{2K}{l}}}$$
 der gegensinnigen Schwingung wurde nur gemessen, da
der Kopplungsgrad $K$ nur aus $T_+$ und $T_-$ bekannt ist.
Damit entfällt die Fehlerrechnung für den Theoriewert.

Der Fehler des Kopplungsgrades $K=\frac{T_+^2-T_-^2}{T_+^2+T_-^2}$ errechnet sich
 zu:
\begin{equation*}
\sigma_K =
\sqrt{\sigma_{T_{+}}^{2} \left(- \frac{2 T_{+} \left(T_{+}^{2} -
T_{-}^{2}\right)}{\left(T_{+}^{2} + T_{-}^{2}\right)^{2}} +
\frac{2 T_{+}}{T_{+}^{2} + T_{-}^{2}}\right)^{2} + \sigma_{T_{-}}^{2}
\left(- \frac{2 T_{-} \left(T_{+}^{2} - T_{-}^{2}\right)}{\left(T_{+}^{2} +
T_{-}^{2}\right)^{2}} - \frac{2 T_{-}}{T_{+}^{2} + T_{-}^{2}}\right)^{2}}.
\end{equation*}

Für den Mittelwert gilt :
\begin{equation*}
\bar{x} = \frac{1}{n}\sum_{i=0}^{n}{x_\text{i}}.
\end{equation*}
womit sich der mittlere Fehler des Mittelwertes zu
$$\text{\Delta} \bar{x} = \frac{\sigma_x}{\sqrt{n}}$$
ergibt.

Die Fehlerformel für die Schwebungsdauer $$T_\text{Schwebung} = \frac
{T_+ \cdot T_-}{T_+-T_-} $$ist
\begin{equation*}
\sigma_{T_{\text{Schwebung}}}=
  \sqrt{\sigma_{T_{+}}^{2} \left(- \frac{T_{+} T_{-}}{\left(T_{+} -
  T_{-}\right)^{2}} + \frac{T_{-}}{T_{+} - T_{-}}\right)^{2} + \sigma_{T_{-}}^{2}
   \left(\frac{T_{+} T_{-}}{\left(T_{+} - T_{-}\right)^{2}} + \frac{T_{+}}{T_{+} - T_{-}}\right)^{2}}.
\end{equation*}
\end{document}
