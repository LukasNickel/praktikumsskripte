\subsection{Pendellänge $l_2 = \SI{50.2 +- 0.2}{\centi\meter}$}
\subsubsection{Ungekoppelt}
Mit der Pendellänge $L_2$ beträgt der Theoriewert für die Schwingungsdauer
$T$ = \SI{1.421 +- 0.007}{\second}. Die Messwerte für die einzelnen Pendel
sind in Tabelle \ref{tab:l2ungekoppelt} zu finden.
Der Mittelwert für das erste Pendel beträgt \SI{1.46220+-0.00032}{\second}, für
das zweite Pendel \SI{1.47400+-0.00032}{\second}.
\begin{table}
 \noindent
 \centering
 \caption{Schwingungsdauern der ungekoppelten
 Pendel für $l_2 = \SI{50.2 +- 0.2}{\centi\meter}$}
 \label{tab:l2ungekoppelt}
 \sisetup{table-format=1.2}
 \begin{tabular}{rr}
 \toprule
  {$T_1$ [\si{\second}]} & {$T_2$[\si{\second}]} \\
 \midrule
    \num{1.420 +- 0.001} & \num{1.458 +- 0.001} \\
    \num{1.482 +- 0.001} & \num{1.486 +- 0.001}\\
    \num{1.488 +- 0.001} & \num{1.464 +- 0.001}\\
    \num{1.452 +- 0.001} & \num{1.476 +- 0.001}\\
    \num{1.458 +- 0.001} & \num{1.494 +- 0.001}\\
    \num{1.448 +- 0.001} & \num{1.498 +- 0.001}\\
    \num{1.482 +- 0.001} & \num{1.466 +- 0.001}\\
    \num{1.446 +- 0.001} & \num{1.486 +- 0.001}\\
    \num{1.466 +- 0.001} & \num{1.466 +- 0.001}\\
    \num{1.480 +- 0.001] & \num{1.446 +- 0.001}\\
 \bottomrule
 \end{tabular}
\end{table}
Die beiden Mittelwerte liegen um <1\% auseinander, auch hier können also zwei
identische Pendel angenommen werden.
Die Schwingungsdauern der gleich- und gegenphasigen Schwingungen
, sind in den Tabellen \ref{tab:l2t+} und \ref{tab:l2t-}
eingetragen.$T_{+,Mittelwert}$ entspricht
\SI{1.482+-0.00032}{\second} auch hier nahezu der Schwingungsdauer
der ungekoppelten Pendel ($\Delta T < \num{1.4}\%$).
\begin{table}
 \noindent
 \centering
 \caption{Schwingungsdauer der gekoppelten
 Pendel bei phasengleicherBewegung}
 \label{tab:l2t+}
 \sisetup{table-format=1.2}
 \begin{tabular}{r}
 \toprule
  {$T_{plus}$ [\si{second}]} \\
 \midrule
    \num{1.464 +- 0.001}\\
    \num{1.464 +- 0.001}\\
    \num{1.488 +- 0.001}\\
    \num{1.500 +- 0.001}\\
    \num{1.482 +- 0.001}\\
    \num{1.442 +- 0.001}\\
    \num{1.480 +- 0.001}\\
    \num{1.500 +- 0.001}\\
    \num{1.504 +- 0.001}\\
    \num{1.492 +- 0.001}\\
 \bottomrule
 \end{tabular}
\end{table}

\begin{table}
 \noindent
 \centering
 \caption{Schwingungsdauer der gekoppelten
 Pendel bei gegenphasiger Bewegung}
 \label{tab:l2t-}
 \sisetup{table-format=1.2}
 \begin{tabular}{r}
 \toprule
  {$T_{plus}$ [\si{second}]} \\
 \midrule
    \num{1.248 +- 0.001}\\
    \num{1.260 +- 0.001}\\
    \num{1.282 +- 0.001}\\
    \num{1.286 +- 0.001}\\
    \num{1.264 +- 0.001}\\
    \num{1.270 +- 0.001}\\
    \num{1.258 +- 0.001}\\
    \num{1.230 +- 0.001}\\
    \num{1.242 +- 0.001}\\
    \num{1.282 +- 0.001}\\
 \bottomrule
 \end{tabular}
\end{table}

Die Schwingungsdauer der gegenphasigen Schwingung ist auch hier
$T_-=\SI{1.262+-0.00032}{\second}$ geringer
als die Schwingungsdauer der gleichphasigen Bewegung
(Vgl. Gleichungen \ref{eqn:omega+} und \ref{eqn:omega-}).

Die Schwingungsdauer der gekoppelten Schwingung liegt bei
$T_{gek} = \SI{1.431+-0.0004}{\second}$. Sie liegt also wie schon bei $l_1$
zwischen der Schwingungsdauer der gleich- und gegenphasigen Schwingung.
Die unterschiedlichen Ungenauigkeiten in Tabelle \ref{tab:l2tgek} kommen dadurch
zustande, dass nach den ersten fünf Messwerten nur noch 3 Periodendauern
aufgenommen wurden.
\begin{table}
 \noindent
 \centering
 \caption{Schwingungsdauer der gekoppelten
 Schwingung}
 \label{tab:l2tgek}
 \sisetup{table-format=1.2}
 \begin{tabular}{r}
 \toprule
  {$T_{gek}$ [\si{second}]} \\
 \midrule
    \num{1.380 +- 0.001}\\
    \num{1.390 +- 0.001}\\
    \num{1.464 +- 0.001}\\
    \num{1.512 +- 0.001}\\
    \num{1.500 +- 0.001}\\
    \num{1.453 +- 0.002}\\
    \num{1.387 +- 0.002}\\
    \num{1.410 +- 0.002}\\
    \num{1.453 +- 0.002}\\
    \num{1.460 +- 0.002}\\
    \num{1.410 +- 0.002}\\
    \num{1.357 +- 0.002}\\
    \num{1.450 +- 0.002}\\
    \num{1.413 +- 0.002}\\
 \bottomrule
 \end{tabular}
\end{table}

\begin{table}
 \noindent
 \centering
 \caption{Schwebungsdauer der gekoppelten
 Schwingung}
 \label{tab:l1tschwebung}
 \sisetup{table-format=1.2}
 \begin{tabular}{r}
 \toprule
  {$T_{gek}$ [\si{second}]} \\
 \midrule
    \num{13.22 +- 0.01}\\
    \num{8.92 +- 0.01}\\
    \num{10.54 +- 0.01}\\
    \num{8.46 +- 0.01}\\
    \num{10.64 +- 0.01}\\
    \num{11.10 +- 0.01}\\
    \num{9.28 +- 0.01}\\
    \num{11.26 +- 0.01}\\
    \num{10.36 +- 0.01}\\
    \num{9.20 +- 0.01}\\
    \num{9.52 +- 0.01}\\
    \num{9.92 +- 0.01}\\
    \num{8.54 +- 0.01}\\
    \num{10.08 +- 0.01}\\
 \bottomrule
 \end{tabular}
\end{table}

Der Mittelwert für die Schwebungsdauer $T_{Schwebung}$
beträgt \SI{10.074 +-0.0032}{\second}.
Aus $T_+$ und $T_-$ ergibt sich gemäß Gleichung \ref{eqn:k} der Kopplungsgrad zu
$K = \num{0.1589 +- 0.00032}$. Mit \ref{eqn:Ts} ergibt sich als Theoriewert
für die Schwebungsdauer $T_S = \SI{8.524 +- 0.018}{\second}$. Dieser Wert liegt
um \num{15.39+-0.0018}\% unter der gemessenen Schwebungsdauer.
