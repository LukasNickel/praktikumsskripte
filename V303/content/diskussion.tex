\section{Diskussion}
\label{sec:Diskussion}
Dieser Abschnitt befasst sich mit möglichen Meßfehlern oder Abweichungen
von der Theorie.

Zunächst erscheint die maximale Spannungsamplitude a dem Funktionsgenerator mit
4,48V relaitv niedrig. Ein Vergleich mit der anderen Praktikumsgruppe lässt
einen Wert erwarten, der eine Größenordnung höher liegt. Nicht ausgeschlossen
ist allerdings, dass die Geräte sich in ihrer Bauweise unterscheiden. In
einem Praktikum aus dem Jahr 2008 tauchte ein Wert von 2,3mV , wobei nichts
über den verwendeteten Lock-In-Verstärker bekannt ist. Die Unterschiede
zwischen versciedenen geräten scheinen sich aber über mehrere Größenordnungen
zu erstrecken.

Beim Mischen der Signale - sowohl mit als auch ohne zugeschalteten
Rauschgenerator - fällt auf, dass die am Phasenschieber eingestellte
Phasenverschiebung $\phi$ von der tatsächlichen verschiebung der Signale
um einen festen Wert abweicht (Wie bereits in der Auswertung erwähnt).
Dies lässt sich durch verschiedene Signallaufzeiten erklären.
