\section{Diskussion}
\label{sec:Diskussion}
Dieser Abschnitt befasst sich mit möglichen Meßfehlern oder Abweichungen
von der Theorie.\\

Zunächst erscheint die maximale Spannungsamplitude an dem Funktionsgenerator mit
4,48V relativ niedrig. Ein Vergleich mit der anderen Praktikumsgruppe lässt
einen Wert erwarten, der eine Größenordnung höher liegt. Nicht ausgeschlossen
ist allerdings, dass die Geräte sich in ihrer Bauweise unterscheiden. In
einem Praktikum aus dem Jahr 2008 tauchte ein Wert von 2,3mV
auf (\cite{Protokoll}), wobei nichts
über den verwendeteten Lock-In-Verstärker bekannt ist. Die Unterschiede
zwischen verschiedenen Geräten scheinen sich also über mehrere Größenordnungen
zu erstrecken.\\

Bei der Betrachtung der Meßwerte von 3.2 und 3.3 (Abb.\ref{fig:plot2low} und
\ref{fig:plot3low})
an dem Tiefpaß - sowohl mit als auch ohne zugeschalteten
Rauschgenerator - fällt auf, dass die am Phasenschieber eingestellte
Phasenverschiebung $\phi$ von der tatsächlichen verschiebung der Signale
um einige Grad abweicht (Wie bereits in der Auswertung erwähnt), ansonsten
aber sehr gut mit der Theorie (Vgl.: Gleichung \ref{eqn:Uout}) harmoniert.
Dies lässt sich durch verschiedene Signallaufzeiten erklären.\\

In 3.4 lässt sich das Abstandsgesetz für die Intensität der LED in
Abb.\ref{fig:plot4} wiederfinden. Die Meßwerte harmonieren ziemlich gut mit der
zu erwartenden $1/r^2$-Abhängigkeit für den Intensitätsabfall im Raum.
