\section{Durchführung}
\label{sec:Durchführung}

Bei dem Versuch kommen ein Lock-In-verstärker, ein Oszilloskop, eine
LED und eine Photodiode zum Einsatz. Der Lock-In-Verstärker beinhaltet
einen Funktionsgenerator  mit 2 Ausgängen, einen Phasenschieber und einen
Rauschgenerator.

\subsection{Funktionsgenerator}
Der erste Teil des Versuches befasst sich mit der Fuktionsweise des Geräts.
Dafür werden die Spannungen an beiden Ausgänge des Funktionsgenerators einzeln
vermessen. Dabei wird festgestellt, dass der linke Ausgang eine regelbare
Spannung generiert, während die Amplitude der Spannung an dem rechten
Ausgang konstant bei 4,48V liegt.

\subsection{Mischen der Signale}
In dem zweiten Versuchsabschnitt werden die beiden generierten Signale gemischt
und integriert. Dafür wird folgende Schaltung aufgebaut:

Abbildung aus Protokoll

Der Rauschgenerator (Noise Gen. auf Abbildung) wird zunächst nicht verwendet.
Die Signalspannung mit einer Frequenz von ~1Khz und einer Amplitude von 1V
wird verstärkt und mit der
Referenzspannung gemischt. Gemessen werden die Ausgangsspannung sowie die
integrierte Ausgangsspannung für verschiedene Phasenverschiebungen $\phi$. Auf
diese 
Weise wird die Phasenabhängigkeit der integrierten Spannung nachgewiesen.

\subsection{Rauschgenerator}
Für diesen Versuchsteil wird der Signalspannung ein Rauschen beigefügt. Dazu
wird der Rauschgenerator eingestellt und ein Rauschen erzeugt, dessen
Größenordnung um eine unter der des Signals liegt. Daraufhin werden alle
Messungen aus dem zweiten Versuhsteil wiederholt um die Werte mit den
unverrauschten Werten zu vergleichen. Die Messungen ergeben, dass das Integral
der gemischten Spannung identisch ist - unabhängig von dem Rauschen.

\subsection{LED und Photodiode}
Im letzten Abschnitt wird die aus den anderen Versuchsteilen bekannte Schaltung
leicht abgewandelt. Anstatt die Signalspannung direkt zu vermessen wird mit ihr
eine LED betrieben. Die Frequenz beträgt dabei 184,4 Hz, die Amplitude ist
maximal mit 4,48V. Das von der LED ausgestrahlte Licht wird von einer Photodiode
im Abstand $r$ vermessen. Die Photodiode ist an den Eingang des Verstärkers
angeschlossen und wird - nachdem es den Bandpass passiert hat - mit der
Referenzspannung gemischt und an die resultierende Spannung an
Oszilloskop und Tiefpaß bestimmt. Da die Signalfrequenz nicht einem Vielfachen
der Netzfrequenz entspricht, wird das durch die Raumbeleuchtung hervorgerufene
Signal am Bandpaß
zum großen Teil ausgefiltert. Dadurch lässt sich die Lichtintensität in
Abhängigkeit von $r$ messen und ein maximaler Abstand für die Detektierung
bestimmen. In diesem Versuch lässt sich allerdings nur nachweisen, dass
dieses $r_{max}$ größer als der maximale Abstand der Apparatur ist.
