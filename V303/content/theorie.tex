\section{Theorie}
\label{sec:Theorie}
\subsection{Zielsetzung}
\label{sec:Zielsetzung}
Ziel des durchgeführten Versuches ist es die Funktionsweise und technischen
Hintergründe des Lock-In-Verstärkers kennenzulernen. Dazu werden einige
Messungen mit verschiedenen Signalen vorgenommen.

\subsection{Theoretische Grundlagen}
\label{sec:Grundlagen}
Der Lock in Verstärker wird hauptsächlich dafür eingesetzt verrauschte Signale
zu messen. Gegenüber einem gewöhnlichen Bandpass bietet ein Lock-In-Verstärker
den Vorteil um einige Größenordnungen höhere Güten zu erzielen.
Der Aufbau eines Lock-In-Verstärkers gestaltet sich wie folgt:
Zunächst wird das Signal auf die Referenzfrequenz $\omega_0$ moduliert.
Danach wird die Signalspannung durch einen Bandpaß geschickt. Dadurch
können bereits Rauschanteile mit einer von der Signalspannung verschiedenen
Frequenz herausgefiltert werden.
Daraufhin wird das Signal mit einer Referenzspannung gemischt. Mit dem
Phasenverschieber lässt sich die Phase zwischen Signal- und Referenzspannung
anpassen.
In nachfolgendem Tiefpass wird die gemischte Spannung integriert und auf diese
Weise von noch vorhandenem Rauschen befreit. Dabei ist zu beachten, dass der
Tiefpass nur für gleiche Frequenzen einen Wert ungleich null liefert. Dies liegt
darin begründet, dass der Tiefpass - so wie er in einem Lock-in-Verstärker
genutzt wird - die Kreuzkorrelation zwischen Referenz- und Signalspannung
berechnet, der Tiefpaß integriert also $U_s_i_g \cdot U_r_e_f$.

Betrachen wir das Beipiel eine sinusförmige Signalspannung $U_s = U_0 sin(\omega
t)$, die moduliert wird durch eine Rechteckspannung $U_r_e_f$. Die Frequenzen
beider Spannungen sind identisch, bezeichnet mit $\omega$,
die Phasenverschiebung ist gegeben durch $\phi$.
Der Tiefpaß integriert $U_s_i_g \cdot U_r_e_f$ über einige Perioden und
unterdrückt dabei die Oberwellen, sodass man eine Gleichspannung erhält, die
proporional zu der Signalspannung ist:
$U_{out} = 2/\pi U_0cos(\phi).

Die Ausgangsspannung hängt also von der Phasenverschiebung ab und ist maximal
für $\phi$ =0.

\cite{sample}
