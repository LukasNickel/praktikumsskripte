\section{Theorie}
\label{sec:Theorie}
\subsection{Zielsetzung}
\label{sec:Zielsetzung}
Ziel des durchgeführten Versuches ist es die Funktionsweise und technischen
Hintergründe des Lock-In-Verstärkers kennenzulernen. Dazu werden einige
Messungen mit verschiedenen Signalen vorgenommen.

\subsection{Theoretische Grundlagen}
\label{sec:Grundlagen}
Der Lock in Verstärker wird hauptsächlich dafür eingesetzt verrauschte Signale
zu messen. Gegenüber einem gewöhnlichen Bandpass bietet ein Lock-In-Verstärker
den Vorteil um einige Größenordnungen höhere Güten zu erzielen.

In dem Lock-In-Verstärker wird eine
Signalspannung $U_{sig}$ sinusförmig
mit der Frequenz $\omega$
moduliert. $U_{sig}$ wird daraufhin mit der Rechteckspannung $U_{ref}$
multipliziert. 
Der Tiefpaß integriert $U_{sig} \times U_{ref}$ über einige Perioden und
unterdrückt dabei die Oberwellen, sodass man eine Gleichspannung erhält, die
proporional zu der Signalspannung
ist :
\begin{equation}
U_{out} = \frac{2}{\pi} U_0 \cos(\phi) = \int U_{sig} \cdot U_{ref}.
\label{eqn:Uout}
\end{equation}
\footnote{Herleitung siehe : \cite[1-3]{Anleitung}}


Die Ausgangsspannung hängt also über einen Cosinus
von der Phasenverschiebung ab und ist maximal
für $\phi$ = 0.
