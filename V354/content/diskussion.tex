\section{Diskussion}
\label{sec:Diskussion}
\subsection{Einordnung des gemessenen Dämpfungswiderstandes}
Der Offset bei der Spannung lässt sich herausrechnen, indem jeweils die Differenz
zwischen zwei aufeinanderfolgenden Werten genommen wird. Auf die Weise erhält man
auch eine Mittelung für die Werte. Der zeitliche Offset kann nach dem
selben Prinzip eliminiert werden, indem der erste Messwert als $t_0$ gewertet
wird und $\Delta t = t_x - t_0$ berechnet wird.
Der Schwingkreis zeigt in guter Näherung das aus der Theorie bekannte
Verhalten der exponentiellen Dämpfung. Der errechnete effektive
Dämpfungswiderstand liegt deutlich höher als der Nennwiderstand.
Ursachen dafür könnten in weiteren Widerständen in den Leitungen und den
Bauteilen liegen.

\subsection{Vergleich der Widerstände der aperiodischen Dämpfung
aus Messung und Theorie}
Der gemessene Wert für $R_\text{ap}$ liegt ca. 30\% unter dem Theoriewert.
Eine Ursache dafür könnte in einer ungenauen Bestimmung des aperiodischen
Grenzfalls liegen. In Abb. \ref{fig:5bergebnis} ist bei genauem Hinsehen ein
minimales Überschwingen zu erkennen. Im Vergleich mit Abb. \ref{fig:über}
lässt sich folgern, dass der tatsächliche Wert für $R_\text{ap}$ etwas höher
als die gemessenen \SI{3120 +- 20}{\ohm} liegt.
Dazu kommen weitere Widerstände in der realen Schaltung, sodass der regelbare
Widerstand nicht den Gesamtwiderstand des Schwingkreises darstellt.
Daher ist es sinnvoll anzunehmen, dass der Dämpfungswiderstand tatsächlich
etwas höher und damit näher am Theoriewert liegt.

\subsection{Diskussion der Güte und Resonanzkurve}
Der Theoriewert, der aus den Parametern des Schwingkreises
berechnet wurde, für die Güte liegt mit $q = 4.309 \pm 0.010$ etwas höher
als die gemessenen Werte. Dabei wurde für den Widerstand der Schaltung allerdings
nur der Nennwiderstand von $R=\SI{509.5+-0.5}{\ohm}$ verwendet. Für den
tatsächlichen Wert muss außerdem der Innenwiderstand des Generators
hinzugerechnet werden (Vgl. \cite{officialmanual}, S.296). Dadurch
sinkt die Güte des Schwingkreises ab (Vgl. Gleichung \ref{eqn:güte1}).
Der Innenwiderstand des Generators ist aus dem Versuch allerdings nicht
bekannt. Es lässt
sich daher nur festhalten, dass die Güte niedriger als \num{4.309 \pm 0.010}
liegen muss. $q=4$ entspricht dabei
einem Gesamtwiderstand von $\approx \SI{550}{\ohm}$.

Bei dem Vergleich der Resonanzkurve mit den Messwerten fällt auf, dass
$U_\text{rel}$
um die Resonanzfrequenz nicht so stark steigt, wie aus der Theorie erwartet.
Dies liegt daran, dass die Güte aus der Theorie etwas höher liegt als die
aus den Messwerten errechnete. Eine Ursache dafür könnte darin liegen, dass
die Güte umgekehrt proportional zu dem Widerstand in der Schaltung ist(siehe
Gleichung \ref{eqn:güte1}).
Da der Widerstand in der realen Schaltung durch weitere Verlustwiderstände neben
dem Nennwiderstand höher als dieser liegt, ergibt sich demnach zwangsläufig
eine geringere Güte und damit ein niedrigeres Maximum der Resonanzkurve.


\subsection{Diskussion der Phasenverschiebung}
Die Theoriekurve deckt sich bis auf die letzten drei Messwerte in sehr guter
Näherung mit den Messwerten. In der Theorie ist die Phasenverschiebung
durch einen Arcustangens gegeben (siehe Gleichung \ref{eqn:phi}) und
kann damit betragsmäßig $\frac{\pi}{2}$ nicht überschreiten.
Da die letzten drei Werte allerdings allesamt größer als $\frac{\pi}{2}$ sind,
kann davon ausgegangen werden, dass ein Fehler bei der Messung gemacht wurde.
Die Werte wurden mittels der Cursor-Funktion des Ozillographens ermittelt, es
ist also anzunehmen, dass dabei die Zeiger falsch positioniert wurden.
